\documentclass[11pt]{article}
\usepackage{fullpage}
\usepackage{fancyhdr}
\usepackage{natbib}
\usepackage{url}

\usepackage{amsmath}
\usepackage{amssymb}

\usepackage[pdftex]{graphicx}

\usepackage{listings}
\usepackage{color}
\lstset{language=Python,
        basicstyle=\footnotesize\ttfamily,
        showspaces=false,
        showstringspaces=false,
        tabsize=2,
        breaklines=false,
        breakatwhitespace=true,
        identifierstyle=\ttfamily,
        keywordstyle=\color[rgb]{0,0,1},
        commentstyle=\color[rgb]{0.133,0.545,0.133},
        stringstyle=\color[rgb]{0.627,0.126,0.941},
    }

% header
\fancyhead{}
\fancyfoot{}
\fancyfoot[C]{\thepage}
\fancyhead[R]{Daniel Foreman-Mackey}
\fancyhead[L]{Extragalactic Astronomy --- Problem Set 2}
\pagestyle{fancy}
\setlength{\headheight}{20pt}
\setlength{\headsep}{10pt}

% shortcuts
\newcommand{\Eq}[1]{Equation (\ref{eq:#1})}
\newcommand{\eq}[1]{Equation (\ref{eq:#1})}
\newcommand{\eqlabel}[1]{\label{eq:#1}}
\newcommand{\Fig}[1]{Figure \ref{fig:#1}}
\newcommand{\fig}[1]{Figure \ref{fig:#1}}
\newcommand{\figlabel}[1]{\label{fig:#1}}
\newcommand{\Tab}[1]{Table \ref{tab:#1}}
\newcommand{\tab}[1]{Table \ref{tab:#1}}
\newcommand{\tablabel}[1]{\label{tab:#1}}

% commands
\newcommand{\sun}{\odot}
\newcommand{\M}{\ensuremath{\mathcal{M}}}
\newcommand{\Msun}{\ensuremath{\M_\sun}}
\newcommand{\dd}{\ensuremath{\,\mathrm{d}}}
\newcommand{\unit}[1]{\ensuremath{\,\mathrm{#1}}}

\begin{document}

\section*{Problem 1}

\section*{Problem 2}

The H$\beta$ line index is defined by \citet{worthey} to be in the wavelength
range 4847.875 \AA--4876.625 \AA. The relevant continuum bands are then
\begin{equation}
    4827.875 \, \mathrm{\AA}-4847.875\,\mathrm{\AA} \quad \mathrm{and} \quad
    4876.625 \, \mathrm{\AA}-4891.625\, \mathrm{\AA} \quad.
\end{equation}
The continuum in the line is defined as the linear interpolation between the
points with wavelengths $\lambda_1 = 4837.875$ \AA\ and $\lambda_2 = 4884.125$
\AA\ and the flux values
\begin{eqnarray}
    f_1 = \frac{1}{20 \, \mathrm{\AA}} \,
        \int_{4827.875} ^{4847.875} F(\lambda) \dd \lambda
    \quad \mathrm{and} \quad
    f_2 = \frac{1}{15 \, \mathrm{\AA}} \,
        \int_{4876.625} ^{4891.625} F(\lambda) \dd \lambda
    \quad.
\end{eqnarray}


\section*{Problem 3}

\paragraph{(a)}

The Salpeter IMF is
\begin{equation}
    \xi (M) = \xi_0 \, M^{-2.35}
\end{equation}
where $\xi_0$ is a normalization constant defined by
\begin{equation}
    M_\mathrm{tot} = 10^6 = \int_{0.08} ^{\infty} \xi_0 \, M^{-1.35} \dd M
        = \frac{0.08^{-0.35} \, \xi_0}{0.35}
\end{equation}
where I chose the minimum mass $M_\mathrm{min} \approx 0.08$ because that is
the minimum mass needed for hydrogen burning. The mass in stars with mass
$>0.5 \, M_\sun$ is then given by
\begin{equation}
    M (>0.5 \, M_\sun) = \int_{0.5} ^\infty \xi (M) \, M \dd M
        = \left ( \frac{0.5}{0.08} \right )^{-0.35} \times 10^6
        = 0.526 \times 10^6 \, M_\sun \quad.
\end{equation}

The \citet{chabrier} IMF is defined as
\begin{equation}
    \xi (\log_{10} M) = \frac{\dd N}{\dd \log_{10} M} = \left \{
    \begin{array}{ll}
        {\displaystyle
            0.086 \, \exp \left ( -\frac{1}{2} \,
            \frac{\left [ \log_{10} M - \log_{10} 0.22 \right ]^2}{0.57^2}
            \right ) \quad,} & M < 1 \\
        {\displaystyle 0.0443 \, M^{-1.3}} \quad, & M > 1
    \end{array} \right . \quad.
\end{equation}
Note the logarithmic binning in mass. To find the normalization of the
function, we must compute
\begin{eqnarray}
    M_\mathrm{tot} & = & N_0 \, \int_{\log 0.08} ^{\infty}
                         M \, \xi (\log M) \dd \log M \\
                   & = & N_0 \, \int_{\log 0.08} ^{\log 1}
                         M \, \xi (\log M) \dd \log M +
                         N_0 \, \int_0 ^\infty M \, \xi(\log M) \dd \log M \\
                   & = & (0.0268 + 0.0641) \, N_0 = 0.0909 \, N_0 \\
\end{eqnarray}
where the first term was solved using Wolfram
Alpha\footnote{\url{http://bit.ly/PMjcei}}. This means that $N_0 =
11 \times 10^6$. Now, the mass in stars with $M > 0.5 \, M_\sun$ is
\begin{equation}
    M (>0.5 \, M_\sun) = (0.0122 + 0.0641) \, N_0 = 0.839 \times 10^6 M_\sun
    \quad.
\end{equation}


\paragraph{(b)} Using equations (5.5) and (5.6) from \citet{BM}, the
main-sequence lifetime of a star with mass $< 2 \, M_\sun$ is
\begin{equation}
    \tau_\mathrm{MS} \sim 10 \, \left (\frac{M}{M_\sun} \right ) \,
        \left ( \frac{L}{L_\sun} \right ) ^{-1}
        \sim \frac{10}{0.75} \, \left ( \frac{M}{M_\sun} \right )^{-3.8}
        \quad.
\end{equation}
This means that the maximum mass of stars still on the main sequence after
10 Gyr is $\sim 0.75^{-1/3.8} \approx 1 \, M_\sun$.

To approximate Figure 5.7 from \citet{BM}, I will use the piecewise linear
function given in \tab{remnant}. The fraction of the stellar mass that is
returned to the ISM is then given by
\begin{eqnarray}
    f & = & \, \int _1 ^{90} \frac{M_\mathrm{initial} -
    M_\mathrm{remnant} (M_\mathrm{initial})}{M_\mathrm{initial}} \,
    \frac{\xi (M_\mathrm{initial})}{M_\mathrm{tot}}
    \dd M_\mathrm{initial} \nonumber \\
    & = & 0.144 \, \sum_i \int_{M_\mathrm{min}^i} ^{M_\mathrm{max}^i}
    \left [x \, (1 - a_i) - b_i \right ] \, x^{-3.35} \dd x \\
    & = & 0.144 \, \sum_i \left \{
    \frac{a_i - 1}{1.35} \, \left [(M_\mathrm{max}^i)^{-1.35} -
        (M_\mathrm{min}^i)^{-1.35} \right ]
    + \frac{b_i}{2.35} \, \left [(M_\mathrm{max}^i)^{-2.35} -
        (M_\mathrm{min}^i)^{-2.35} \right ]
    \right \} \nonumber
\end{eqnarray}
where the sum is over the piecewise components and $a_i$ and $b_i$ are the
slope and intercept (respectively) in component $i$. Using the approximation
in \tab{remnant}, the result is
\begin{equation}
    f \approx 0.074 = 7.4 \%
\end{equation}
of the mass is returned to the ISM after 10 Gyr.

\begin{table}

    \centering

    \begin{tabular}{c c}

        $M_\mathrm{inital}$ [$M_\sun$] & $M_\mathrm{remnant}$ [$M_\sun$] \\

        \hline

        1  & 0.5 \\
        6  & 0.7 \\
        10 & 1.3 \\
        60 & 1.4 \\
        90 & 3.0 \\

    \end{tabular}

    \caption{The approximate remnant-mass-initial-mass relation.
        \tablabel{remnant}}

\end{table}


\section*{Problem 4}

Equation (5.6) in \citet{BM} provides a rough approximation of the main
sequence lifetime for stars as a function of luminosity and initial mass.
This scaling relation combined with Equation (5.5) gives an approximation
of the main sequence lifetime as a function of initial mass alone
\begin{equation}
    \tau_\mathrm{approx} \sim \frac{10}{0.75} \, \left ( \frac{M}{M_\sun}
    \right )^{-3.8}
    \quad.
\end{equation}
\tab{mslt} shows this approximate main sequence lifetime as a function of
initial mass for the masses shown in Figure (5.4) in \citet{BM}. \tab{mslt}
also lists the main sequence lifetimes from Figure (5.4) in \citet{BM} and
there is $< 1\%$ discrepancy at all masses.


\begin{table}

    \centering

    \begin{tabular}{c c c}

        $M_\mathrm{inital}$ [$M_\sun$] & $\tau_\mathrm{approx}$ [Gyr] &
        $\tau_\mathrm{MS}$ [Gyr] \\

        \hline

        0.9 & 19.9 & 17.8 \\
        1.0 & 13.3 & 12.7 \\
        1.5 & 2.86 & 2.7  \\
        2.0 & 0.96 & 1.0  \\

    \end{tabular}

    \caption{The approximate main sequence lifetime as a function of initial
    mass. The third column shows the values read off Figure (5.4) in
    \citet{BM}. \tablabel{mslt}}

\end{table}


\newpage

\begin{thebibliography}{}
\raggedright


\bibitem[Chabrier(2003)]{chabrier} Chabrier, G.\ 2003, PASP,
    115, 763

\bibitem[Binney \& Merrifield(1998)]{BM}
    Binney, J., \& Merrifield, M.\ 1998, \emph{Galactic Astronomy},
    Princeton University Press, 1998

\bibitem[Worthey et al.(1994)]{worthey} Worthey, G., Faber,
    S.~M., Gonzalez, J.~J., \& Burstein, D.\ 1994, ApJS, 94, 687


\end{thebibliography}

\end{document}

